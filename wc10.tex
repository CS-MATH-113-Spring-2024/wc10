\documentclass[a4paper]{exam}

\usepackage{amsfonts,amsmath,amsthm}
\usepackage[a4paper]{geometry}
\usepackage{xcolor}

\usepackage{draftwatermark}
\SetWatermarkText{Sample Solution}
\SetWatermarkScale{3}
\SetWatermarkLightness{.95}
\printanswers

\newcommand\N{\ensuremath{\mathbb{N}}}
\newcommand\union{\cup}
\newcommand\interx{\cap}

\header{CS/MATH 113}{WC10: Set Cardinality}{Spring 2024}
\footer{}{Page \thepage\ of \numpages}{}
\runningheadrule
\runningfootrule

\printanswers

\qformat{{\large\bf \thequestion. \thequestiontitle}\hfill(\thepoints)}
\boxedpoints

\title{Weekly Challenge 10: Set Cardinality}
\author{CS/MATH 113 Discrete Mathematics}
\date{Spring 2024}

\begin{document}
\maketitle

\begin{questions}

\titledquestion{Cardinality and Set Operations}[0]
  This ungraded problem provides the background for the next, graded problem for which you will require one or more of the results below. Find and go over the proofs of these results. No submission is needed for this problem but the results will prove useful for the next problem.
  
  \begin{parts}
  \part The union of countably many countable sets is countable.
  \part The superset of an uncountable set is uncountable.
  \part The powerset of a countable set is uncountable.
  \end{parts}


\titledquestion{Fibonacci Unchained}[10]

  Consider an infinite matrix, $A$, in which the entry in the $i$-th row and $j$-th column is defined as follows.
  \[
    a[i,j] = a[i,j-1] + a[i,j-2], \quad a[i,1] = i, a[i,2] = i+1, \quad i\geq 1, j\geq 1.
  \]
  Consider a matrix, $B$, obtained from $A$ as follows.
  \[
    b[i,j]= a[i,i]
  \]
  Argue whether the number of entries in $B$ is countable.
  
  \begin{solution}
    We notice that the number of entries in $B$ does not depend on how each entry is computed.\\
    Also, the number of entries in $B$ is the same as in $A$.\\
    Each row in $A$ is a Fibonacci sequence with unique starting conditions.\\
    Therefore, each row is $A$ has countably many terms.\\
    Let us define a \textit{row-set} for each row, i.e. the set containing the terms of the row.\\
    Each row-set is countable.\\
    There are countably many rows and hence row-sets in $A$.\\
    The total number of terms in $A$ can be seen as a union of the row-sets.\\
    This is a countable union of countable sets.\\
    From the previous question, the resulting set is countable.\\
    Therefore, the number of terms in $A$ is countable.\\
    Therefore, the number of terms in $B$ is countable.
  \end{solution}

\end{questions}

\end{document}
%%% Local Variables:
%%% mode: latex
%%% TeX-master: t
%%% End:
